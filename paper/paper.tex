\documentclass[10pt,a4paper,twocolumn]{article}
\RequirePackage[italian]{babel}
\usepackage[utf8]{inputenc}
\usepackage{amsmath}
\usepackage{amsfonts}
\usepackage{amssymb}
\usepackage{graphicx}

\author{M. Faretra, G. Marini, A. Martinelli}
\title{Raccolta accurata di fatti da testo in linguaggio naturale di Wikipedia}
\begin{document}
	
\maketitle
		
\section*{RIASSUNTO}
		
Molti approcci sono stati utilizzati per estrarre informazione da Wikipedia sotto forma di fatti (entità, relazione, entità) per il popolamento di Knowledge Graphs, in particolare sfruttando le informazione contenute nelle sue infoboxes. Tuttavia queste strutture dati riportano solo una piccola parte delle informazioni contenute negli articoli. Infatti nel testo libero si concentra la maggior parte delle relazioni estraibili, tuttavia la rilevazione di essi risulta più problematica trovandosi all'interno di testo libero in linguaggio naturale. In questo lavoro si cerca di quantificare il numero di relazioni estratte da questi articoli con il supporto di un KG già popolato per aumentarne la conoscenza. In particolare, la nostra valutazione è stata effettuata utilizzando DBpedia, un KG gratuito, che ci ha portato a... (to be implemented, forse spiegare come è rappresentata la conoscenza con le triple)
		
\section{INTRODUZIONE} 
		
L'incremento dei Knowledge Graph in questi ultimi tempi è stata di particolare interesse scientifico e hanno evidenziato la limitatezza e la mancanza di informazioni in essi presente, dovuta in alcuni casi a entità inserite automaticamente senza particolari relazioni, in altri alla limitata conoscenza reperibile dagli infoboxes.
		
Ad esempio il progetto DBpedia estrae informazioni da più di 125 edizioni in linguaggi differenti di Wikipedia. La più grande base di conoscenza è estratta dalla versione inglese e consiste in più di 580 milioni di fatti che descrivono 38 milioni di cose. Il progetto DBpedia mappa le infoboxes di Wikipedia da 28 edizioni in linguaggi differenti in una singola ontologia condivisa consistente di circa 685 classi e 2795 proprietà. 
		
Considerando questi numeri e la loro fonte (infoboxes), la conoscenza estraibile dal testo naturale potrebbe essere decisamente più consistente e aumentare di molto il Knowledge Graph. L'approccio utilizzato va a scalare sfruttando i fatti contenuti nel KG stesso, è quindi dipendente anche da essi, ed in particolare dalla loro qualità: le triple sono etichettate come fidate e non fidate e con esse abbiamo effettuato misurazioni differenti, considerando le relazioni candidate estratte dalle triple appartenenti all'una o all'altra categoria.
		
L'utilizzo di Wikipedia è ampiamente diffuso tra i vari KG implementati data la grande affidabilità che ormai garantisce, tuttavia le infoboxes fino a pochi anni fa erano ancora molto poco diffuse e solo nell'ultimo decennio esse si trovano in più della metà degli articoli.

Se invece si va a considerare la conoscenza presente nel testo libero, si può immaginare che da esso (ovviamente presente in ogni articolo) si possa estrarre informazione non presente nel KG poichè esso riporta una serie di relazioni che probabilmente non sono presenti nell'infobox, ad esempio riguardanti entità che non sono il soggetto dell'articolo.

Il nostro approccio al problema considera pattern del tipo [entità] frase [entità], ad esempio: "Antonio è poco propenso alla Scrittura" mette in relazione le due entità "Antonio" e "Scrittura" utilizzando la frase "è poco propenso alla" che descrive un'istanza della relazione.
In questo articolo descriviamo l'approccio di estrazione di conoscenza da testo libero di Wikipedia, e i risultati in termini di aumento dei fatti presenti in DBpedia, che usiamo anche come garante dell'effettiva utilità della frase per esprimere una determinata relazione.

(sommaria spiegazione del processo e qualche numero di risultati)
Il resto dell'articolo è organizzato in vari capitoli: nel Capitolo 2 vengono analizzate in maggior dettaglio le risorse utilizzate; nel Capitolo 3 si presenta il nostro approccio; nel Capitolo 4 vengono mostrati i nostri risultati; nel Capitolo 5 si trovano cenni a lavori correlati e su cui ci siamo basati (se vogliamo scriverci qualche cagata su Lector da cui siamo partiti); infine, nel Capitolo 6 vengono presentate le conclusioni sul lavoro effettuato.

\section{RISORSE}
\subsection*{\textit{Wikipedia.}}

Wikipedia è un'enciclopedia online libera e collaborativa, che attualmente comprende circa 5.3 milioni di articoli nella sua versione inglese. Questa modalità di collaborazione garantisce una grande qualità e affidabilità sulle informazione ed anche una certa omogeneità nell'esprimere determinati concetti che va a facilitare l'estrazione di fatti.
 
Ogni articolo in Wikipedia fa riferimento ad una entità principale, che può rappresentare una persona, un luogo, un oggetto, un fatto, ecc..., identificata da un identificatore. Nel testo le informazioni sono codificate in linguaggio naturale, inoltre le entità secondarie eventualmente presenti in un determinato articolo sono rappresentate usando dei \textit{wikilinks}, una sintassi specifica di Wikipedia per evidenziare un particolare concetto e offrire un link all'articolo che lo descrive per avere un quadro più generale sulla questione.

Nel nostro particolare caso le frasi di Wikipedia su cui lavorare ci sono state fornite dal docente poichè il progetto è stato svolto in ambito universitario, nella forma di sentenze con le entità già etichettate in una particolare modalità:\\

(aggiungere immagine di esempio)

\subsection*{\textit{DBpedia.}}

DBpedia è frutto di un lavoro di collaborazione da parte di una moltitudine di utenti per estrarre informazioni strutturate da Wikipedia stessa e renderle disponibili sul Web.

Ogni cosa in DBpedia è identificata da un URI del tipo:\\\textbf{http://dbpedia.org/resource/Name}\\ dove "Name" è preso dall'URL del relativo articolo di Wikipedia, che ha la forma:\\\textbf{http://dbpedia.org/resource/Name}.\\
In questo modo ogni risorsa è legata direttamente ad un articolo di Wikipedia. I dati sono suddivisi in dataset diversi ed in particolare quelli che abbiamo preso in considerazione per il progetto sono due:
\begin{itemize}
\item quello relativo ai tipi delle entità, ovvero una serie di coppie "URL entità - URL tipo", che associa ad esempio all'entità "Barack Obama" il tipo "Persona";
\item un dataset relativo agli schema e ai collegamenti tra entità e relazioni utilizzati da DBpedia che analizzeremo in seguito
\end{itemize}

\section{APPROCCIO}

Il dataset iniziale di frasi, come detto precedentemente, ci è stato fornito con le entità etichettate. Inizialmente abbiamo provato ad applicare un approccio euristico per il riconoscimento di frasi di tipo lista, basato su un lavoro precedentemente svolto da alcuni colleghi, che si è rivelato non scalabile. Si è quindi preferito perdere parte dell'informazione in esse presente favorendo la velocità, data la grande mole di dati.
Il procedimenti si svolge quindi in diversi passaggi:
\begin{enumerate}
\item Verifica della relazionalità della frase in questione;
\item Etichettatura delle triple "presenti" e "non presenti";
\item Scoring delle frasi;
\item Estrazione di informazione.
\end{enumerate}

\subsection{Verifica delle frasi}
Il dump iniziale delle frasi ci è stato fornito dal professore con le entità delimitate da parentesi quadre e arricchite con l'id relativo su \textit{Freebase}, superfluo per questo lavoro. Il dato all'interno di esse ci fornisce il \textit{wikid} che ci è stato utile per reperire l'entità anche su DBpedia, che usa proprio questi identificatori per salvarle, poiché basato proprio su Wikipedia. In alcuni casi questi identificatori fanno riferimento ad un redirect che punta all'effettivo wikid che abbiamo reperito tramite una mappatura offerta da un altro file di input fornito dal docente.

Abbiamo quindi suddiviso le frasi iniziali in una lista di triple comprendenti tutte le coppie di entità e il testo compreso tra esse, perdendo parte dell'informazione presente ad esempio in frasi di tipo lista nelle quali il soggetto andrebbe collegato con le varie entità in relazione con esso utilizzando il pezzo di frase che lo collega con la prima entità. (eventuale immagine).

Le frasi tra le entità sono state quindi passate ad un filtro per trovare quelle che esprimono una relazione. Questa operazione snellisce di molto il dataset iniziale eliminando la maggior parte degli spezzoni di testo tra entità poiché il filtro è molto restrittivo, garantendo dall'altra parte un output di frasi che quasi sicuramente collega in maniera relazionale due entità.

\subsection{Etichettatura}
A questo punto, abbiamo considerato le entità di ogni tripla per verificarne l'effettiva presenza all'interno di DBpedia.

Le triple "etichettate" rappresentano fatti già presenti nel KG e sono utili per ricavare una mappatura tra le frasi che legano le entità e le relazioni ad esse accomunabili.

Quelle non etichettate sono invece candidate a rappresentare quella parte di conoscenza che il nostro lavoro va a reperire in più, ampliando l'informazione della base di conoscenza.

\subsection{Scoring}

Ogni frase di quelle non etichettate è associata ad una o più relazione proveniente da DBpedia, ma ovviamente queste corrispondenze potrebbero essere non veritiere o valide solo in alcune occorrenze. Abbiamo quindi assegnato ad ogni frase un punteggio per l'appartenenza o meno ad una relazione considerando il numero di volte che esse sono associate. In questo modo abbiamo cercato di penalizzare frasi che esprimono concetti molto generici poiché non danno molta fiducia sulla correttezza dell'associazione.

Il punteggio viene assegnato tenendo conto del numero di volte che la frase è collegata alla determinata relazione ( $c(p,r_i)$ ), al conteggio totale delle sue occorrenze in tutte le relazioni ( $\sum_{j \in R} c(p,r_j)$ ) e al numero di relazioni a cui è associata ( $c(R|p)$ ) secondo la seguente formula:
\[score(p,r_i)=c(p,r_i)\cdot\frac{c(p,r_i)}{\sum_{j \in R}c(p,r_j)}\cdot \frac{1}{c(R|p)} \]
In questo modo una frase ottiene un punteggio maggiore quanto più compare associata alla relazione in questione rispetto al numero totale di occorrenze, penalizzando le frasi in base al numero di relazioni a cui sono associate, poiché si tratta di forme con molta generalità e utilizzate per esprimere un grande insieme di relazioni diverse. 

\subsection{Estrazione dell'informazione}


\end{document}